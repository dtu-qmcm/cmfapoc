\documentclass{article}

\usepackage{arxiv}

\usepackage[utf8]{inputenc} % allow utf-8 input
\usepackage[T1]{fontenc}    % use 8-bit T1 fonts
\usepackage{hyperref}       % hyperlinks
\usepackage{url}            % simple URL typesetting
\usepackage{booktabs}       % professional-quality tables
\usepackage{amsfonts}       % blackboard math symbols
\usepackage{nicefrac}       % compact symbols for 1/2, etc.
\usepackage{microtype}      % microtypography
\usepackage{cleveref}       % smart cross-referencing
\usepackage{lipsum}         % Can be removed after putting your text content
\usepackage{graphicx}
\usepackage{natbib}
\usepackage{doi}

\def\tightlist{}

\title{Compositional Metabolic Flux Analysis}

\date{}

\usepackage{authblk}
\renewcommand\Authfont{\bfseries}
\setlength{\affilsep}{0em}
\newbox{\orcid}\sbox{\orcid}{\includegraphics[scale=0.06]{orcid.pdf}} 


\author[1]{
  \href{asdfasdef}{\usebox{\orcid}\hspace{1mm}Teddy Groves}
}
\author[1]{
  \href{asdfasdef}{\usebox{\orcid}\hspace{1mm}Te Chen}
}
\author[1]{
  \href{asdfasdef}{\usebox{\orcid}\hspace{1mm}Sergi Muyo Abad}
}
\author[1]{
  \href{asdfasdef}{\usebox{\orcid}\hspace{1mm}Nicholas Luke Cowie}
}
\author[1]{
  \href{asdfasdef}{\usebox{\orcid}\hspace{1mm}Daria Volkova}
}
\author[2]{
  \href{asdfasdef}{\usebox{\orcid}\hspace{1mm}Christian Brinch}
}
\author[1,3]{
  \href{asdfasdef}{\usebox{\orcid}\hspace{1mm}Lars Keld Nielsen}
}

\affil[1]{The Novo Nordisk Center for Biosustainability, DTU, Kongens
Lyngby, Denmark}
\affil[2]{National Food Institute, DTU, Kongens Lyngby, Denmark}
\affil[3]{Australian Institute for Bioengineering and Nanotechnology
(AIBN), The University of Queensland, St Lucia 4067, Australia}


% \renewenvironment{abstract}
%  {\small
%   \begin{center}
%   \bfseries \abstractname\vspace{-.5em}\vspace{0pt}
%   \end{center}
%   \list{}{
%     \setlength{\leftmargin}{1.5cm}%
%     \setlength{\rightmargin}{\leftmargin}%
%   }%
%   \item\relax}
% {\endlist}

\hypersetup{
	pdftitle={Compositional Metabolic Flux Analysis},
	pdfauthor={ Teddy Groves,   Te Chen,   Sergi Muyo Abad,   Nicholas Luke
Cowie,   Daria Volkova,   Christian Brinch,   Lars Keld Nielsen,   },
    colorlinks=true,
    linkcolor=black,
    filecolor=black,
    urlcolor=black,
	citecolor=black
}
\begin{document}
\maketitle

\begin{abstract}
	Metabolic Flux Analysis aims to infer the values of metabolic fluxes
from measurements of isotope labelling distributions. Since these
distributions are positive, sum-constrained and relatively
low-dimensional, we argue that they should be analysed using specialised
methods that target compositional data. We illustrate our argument using
a simple pedagogical example, then show how compositional analysis leads
to improved results on a typical dataset.
\end{abstract}



\section{Introduction}\label{introduction}

Labelling-based Metabolic flux analysis is the study of

\subsection{Previous work}\label{previous-work}

This section briefly reviews previous work in labelling-based metabolic
flux analysis. For more detailed review papers see XXXXX

\subsubsection{Experimental methods}\label{experimental-methods}

\subsubsection{The forward problem}\label{the-forward-problem}

\begin{itemize}
\tightlist
\item
  Cumomers
\item
  EMU
\end{itemize}

\subsubsection{Software}\label{software}

\begin{itemize}
\tightlist
\item
  OpenFlux
\item
  Freeflux
\item
  INCA
\item
  13CFlux
\item
  \ldots{}
\end{itemize}

\subsubsection{Bayesian 13C MFA}\label{bayesian-13c-mfa}

\subsection{Problem statement}\label{problem-statement}

The topic of how to statistically model isotope labelling pattern
measurements has received relatively little attention in the development
of metabolic flux analysis. Most presentations and software applications
quantify the discrepancy between a species's measured and predicted
isotope labelling distribution using a Euclidean distance, and advocate
choosing a flux configuration whose labelling pattern minimises this
distance, possibly with per-species and/or per-isotope-equivalence-class
weights. This is equivalent to using maximum likelihood estimation,
where the likelihood is given by an indepedent normal distribution
centered on the predictedc labelling distribution, with error standard
deviations determined by the weights, i.e., for each species s,

\[
y_s \sim N(\hat{y_s}, \sigma_s)
\]

where \(y_s\) is the observed labelling distribution for species \(s\),
\(\hat{y_s}\) is the predicted labelling distribution and \(sigma_s\) is
a vector of standard deviations.

There are two key reasons why this approach is flawed in the case where
\(y_s\) and \(\hat{y_s}\) are compositions. First, the Euclidean
distance is inappropriate for measuring discrepancies between
compositions. Second, the use of an independent error model neglects the
fact that composition components are intrinsically correlated. This
issue is especially pronounced in the case where there are relatively
few composition components.

\subsection{Isotopes, isotopologues and mass
isotopologues}\label{isotopes-isotopologues-and-mass-isotopologues}

Isotopes are atoms whose nuclei have the same number of protons but
different numbers of neutrons. Isotopes instantiate the same element and
have very similar chemical properties, but have different atomic masses
and physical properties. For example, Carbon has three naturally
occurring isotopes: 12C, 13C and 14C, with respective atomic masses 12,
13 and 14. 14C occurs in negligible quantities, and the natural ratio of
12C to 13C is known, making carbon suitable for isotope labelling
experiments where 12C is artificially replaced with 13C.

Isotopologues are forms of a compound that differ only by substitution
of isotopes. For example, {[}1-13C{]} glucose, {[}U-13C{]} glucose and
{[}2-13C{]} glucose are isotopologues that differ only in the isotopes
of the carbon atoms in positions 1 and 2. In general, for a compound
with \(A\) occurrences of an atom with \(I\) isotopes, there are \(I^A\)
corresponding isotopologues. For example, glucose has six carbon atoms:
assuming only 12C and 13C isotopes are present, there are \(2^6\) carbon
isotopologues.

A mass isotopologue is an equivalence class of isotopologues that share
the same atomic mass. For example, {[}1-13C{]} glucose and {[}2-13C{]}
glucose each have five 12C atoms and one 13C atom and therefore belong
to the glucose mass isotopologue \(M_1\) with atomic mass 181.15 g/mol.
Mass isotopologues are important because measurements can often
distinguish between mass isotopologues, but not between isotopologues
with the same atomic mass.

\subsection{13C labelling experiments}\label{c-labelling-experiments}

\subsection{13C Metabolic Flux
Analysis}\label{c-metabolic-flux-analysis}

13C MFA considers a known metabolic network consisting of \(M\)
compounds and \(N\) reactions with stoichiometric coefficients
\(S\in\mathbb{R}^{M\times N}\) representing the amount of each compound
consumed and produced by each reaction, plus an atom transition map for
each reaction. The atom transition map for a reaction specifies in what
order the potentially-labelled atoms occur in each of the reaction's
substrates and products.

The remaining input for 13C MFA is as follows:

\begin{itemize}
\item
  Known isotope proportions for some compounds, typically the feed.
\item
  Measured fluxes for some reactions, possibly with known measurement
  error.
\item
  Measured mass isotopologue proportions for some compounds, possibly
  with known measurement error.
\end{itemize}

The task of inferring the label pattern corresponding to a known flux
assignment is known as the ``forward problem''. {[}REFERENCE{]} shows
how, assuming that the network is in a metabolic and isotopic steady
state, so that neither the concentrations of the compounds nor the
distributions of isotopologues are changing, it is possible to calculate
the isotopologue distribution for each compound given a known flux; in
this way one can calculate the labelling pattern \(r(v)\) corresponding
to any flux assignment \(v\).

Unfortunately, solving the forward problem in terms of isotopologue is
of limited use for real applications due to the prohibitively large
number of isotopomers that need to be considered. As a result of this
difficulty there has been considerable interest in more concise
representations of the forward problem {[}REFERNECES{]}. Below
{[}INTERNAL REFERENCE{]} we consider in detail the ``elementary
metabolite unit'' representation introduced in
\citep{antoniewiczElementaryMetaboliteUnits2007}.

The inverse problem of inferring steady state fluxes from measured
isotopologue distributions can be solved using a statistical model that
links these measurements with latent parameters representing flux
configurations. In general, such a model specifies the probability
\(p(r_{obs}\mid r(v))\) of observing labelling pattern \(r_{obs}\) given
a true flux assignment \(v\) and true labelling pattern \(r(v)\). For
example, assuming a linear model, or equivalently optimising \(v\) by
least squares, yields the following relationship:

\[
r_{obs} \sim N(r(v), \Sigma)
\]

\subsection{The Elementary Metabolite Unit
representation}\label{the-elementary-metabolite-unit-representation}

\subsection{Compositional Regression}\label{compositional-regression}

Compositional data is data that is subject to a unit-sum constraint. For
example, a compositional dataset might record the amount of fat, protein
and other ingredients in some blocks of butter as proportions of the
total mass of each block. These proportions are constrained to sum to
exactly one.

It is well known that, in general, applying non-compositional data
analysis methods to compositional data is dangerous because these
methods can easily misinterpret constraint-induced correlations
\citep[Ch. 3]{aitchisonjStatisticalAnalysisCompositional}.

Compositional regression methods employ constrained measurement
distributions to analyse compositional data, allowing induced
correlations to be accounted for naturally. Examples of such
distributions include the logistic-normal and Dirichlet distributions
\citep[Ch. 3]{aitchisonjStatisticalAnalysisCompositional} among others.

Compositional regression methods are appropriate for 13C MFA because
mass isotopologue distribution vectors are compositional. We therefore
considered it likely that the standard practice of applying
non-compositional statistical analysis to such data would produce
incorrect results.

\subsection{Existing solutions}\label{existing-solutions}

Existing implementations of 13C MFA include:

\begin{itemize}
\tightlist
\item
  INCA
\item
  13CFLUX2
\item
  Metran
\item
  OpenFlux(2)
\item
  FluxPyt
\item
  mfapy
\item
  Sysmetab
\item
  iso2flux
\item
  Flux-P
\item
  WUFlux
\item
  OpenMebius
\item
  influx\_s
\end{itemize}

See \citep{daiUnderstandingMetabolismFlux2017},
\citep{falcoMetabolicFluxAnalysis2022}for reviews of available software
implementing 13C MFA. We wish to note several limitations of the
currently available software:

\begin{itemize}
\tightlist
\item
  There is no previous implementation of compositional regression
  analysis in the context of 13C MFA; all previous implementations apply
  a linear model either explicitly as in \citep[Eq.
  3]{theorellBeCertainUncertainty2017} or more commonly implicitly
  through the use of least-squares optimisation.
\item
  The only software implementing Bayesian 13C MFA is proprietary.
\end{itemize}


\bibliographystyle{unsrtnat}
\bibliography{refs.bib}
\end{document}
