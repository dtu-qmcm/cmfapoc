\documentclass{article}

\usepackage{arxiv}

\usepackage[utf8]{inputenc} % allow utf-8 input
\usepackage[T1]{fontenc}    % use 8-bit T1 fonts
\usepackage{hyperref}       % hyperlinks
\usepackage{url}            % simple URL typesetting
\usepackage{booktabs}       % professional-quality tables
\usepackage{amsfonts}       % blackboard math symbols
\usepackage{nicefrac}       % compact symbols for 1/2, etc.
\usepackage{microtype}      % microtypography
\usepackage{cleveref}       % smart cross-referencing
\usepackage{lipsum}         % Can be removed after putting your text content
\usepackage{graphicx}
\usepackage{natbib}
\usepackage{doi}

\def\tightlist{}

\title{Compositional Metabolic Flux Analysis}

\date{}

\usepackage{authblk}
\renewcommand\Authfont{\bfseries}
\setlength{\affilsep}{0em}
\newbox{\orcid}\sbox{\orcid}{\includegraphics[scale=0.06]{orcid.pdf}} 


\author[1]{
  \href{asdfasdef}{\usebox{\orcid}\hspace{1mm}Teddy Groves}
}
\author[1]{
  \href{asdfasdef}{\usebox{\orcid}\hspace{1mm}Te Chen}
}
\author[1]{
  \href{asdfasdef}{\usebox{\orcid}\hspace{1mm}Sergi Muyo Abad}
}
\author[1]{
  \href{asdfasdef}{\usebox{\orcid}\hspace{1mm}Nicholas Luke Cowie}
}
\author[1]{
  \href{asdfasdef}{\usebox{\orcid}\hspace{1mm}Daria Volkova}
}
\author[2]{
  \href{asdfasdef}{\usebox{\orcid}\hspace{1mm}Christian Brinch}
}
\author[1,3]{
  \href{asdfasdef}{\usebox{\orcid}\hspace{1mm}Lars Keld Nielsen}
}

\affil[1]{The Novo Nordisk Center for Biosustainability, DTU, Kongens
Lyngby, Denmark}
\affil[2]{National Food Institute, DTU, Kongens Lyngby, Denmark}
\affil[3]{Australian Institute for Bioengineering and Nanotechnology
(AIBN), The University of Queensland, St Lucia 4067, Australia}


% \renewenvironment{abstract}
%  {\small
%   \begin{center}
%   \bfseries \abstractname\vspace{-.5em}\vspace{0pt}
%   \end{center}
%   \list{}{
%     \setlength{\leftmargin}{1.5cm}%
%     \setlength{\rightmargin}{\leftmargin}%
%   }%
%   \item\relax}
% {\endlist}

\hypersetup{
	pdftitle={Compositional Metabolic Flux Analysis},
	pdfauthor={ Teddy Groves,   Te Chen,   Sergi Muyo Abad,   Nicholas Luke
Cowie,   Daria Volkova,   Christian Brinch,   Lars Keld Nielsen,   },
    colorlinks=true,
    linkcolor=black,
    filecolor=black,
    urlcolor=black,
}
\begin{document}
\maketitle

\begin{abstract}
	Metabolic Flux Analysis aims to infer the values of metabolic fluxes
from measurements of isotope labelling distributions. Since these
distributions are positive, sum-constrained and relatively
low-dimensional, we argue that they should be analysed using specialised
methods that target compositional data. We illustrate our argument using
a simple pedagogical example, then show how compositional analysis leads
to improved results on a typical dataset.
\end{abstract}


\section{Introduction}\label{introduction}

Metabolic flux analysis is the study of

\subsection{Previous work}\label{previous-work}

This section briefly reviews previous work in labelling-based metabolic
flux analysis. For more detailed review papers see XXXXX

\subsubsection{Experimental methods}\label{experimental-methods}

\subsubsection{The forward problem}\label{the-forward-problem}

\begin{itemize}
\tightlist
\item
  Cumomers
\item
  EMU
\end{itemize}

\subsubsection{Software}\label{software}

\begin{itemize}
\tightlist
\item
  OpenFlux
\item
  Freeflux
\item
  INCA
\item
  13CFlux
\item
  \ldots{}
\end{itemize}

\subsubsection{Bayesian 13C MFA}\label{bayesian-13c-mfa}

\subsection{Problem statement}\label{problem-statement}

The topic of how to statistically model isotope labelling pattern
measurements has received relatively little attention in the development
of metabolic flux analysis. Most presentations and software applications
quantify the discrepancy between a species's measured and predicted
isotope labelling distribution using a Euclidean distance, and advocate
choosing a flux configuration whose labelling pattern minimises this
distance, possibly with per-species and/or per-isotope-equivalence-class
weights. This is equivalent to using maximum likelihood estimation,
where the likelihood is given by an indepedent normal distribution
centered on the predictedc labelling distribution, with error standard
deviations determined by the weights, i.e., for each species s,

\bibliographystyle{unsrtnat}
\bibliography{refs.bib}  %%% Uncomment this line and comment out the ``thebibliography'' section below to use the external .bib file (using bibtex) .

\end{document}
